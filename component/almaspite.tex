\project project
\product product

\startcomponent almaspite

\startsubsection[title={Almás pite},bookmark={Almás pite},reference={almaspite}]
\startitemize[R]
  \startitem {\tt \unit{300 gram} liszt, \unit{150 gram} vaj, 2 db tojássárgája, \unit{80 gram} cukor, \unit{1 deciliter} tejföl, só, \unit{1,2 kilogram} alma, \unit{50 gram} zsemlemorzsa, 1db citrom, őrölt fahéj, vaníliás porcukor} \crlf
             A lisztet, vajat, \unit{80 gram} cukrot, csipet reszelt citrom héjat gyúródeszkán összemorzsoljuk. Hozzáadjuk a tejfölt, a tojássárgát, és jól összedolgozzuk. Rövid ideig pihentetjük. A tészta felét ceruza vastagságúra kinyújtjuk, tepsibe rakjuk. Villával megszurkáljuk és kissé átsütjük. Ezután kevés morzsával meghintjük, ráhalmozzuk a lehámozott, szeletekre vágott almát. \crlf
             Őrölt fahéjjal, cukorral megszórjuk. A tészta másik, vékonyra nyújtott felével befedjük. Tojással bekenjük, néhány helyen megszurkáljuk. Mérsékelten meleg sütőben aranybarnára sütjük. \unit{190 degree celsius} \unit{30 arcminute} Ha kissé kihűlt, kockákra vágjuk, és vaníliás porcukorral meghintjük. \stopitem
  \startitem {\tt \unit{500 gram} liszt, \unit{280 gram} vaj vagy zsír, 1 egész tojás, 1 tojás sárgája, 4 ek porcukor, só, bor} \crlf
             Reviczky Töltelék: {\tt \unit{1 kilogram} reszelt alma, vaníliás cukor, mazsola, dió, fahéj, citromhéj}. \unit{190 degree celsius} \unit{30 arcminute} \stopitem
\stopitemize
\stopsubsection

\stopcomponent
