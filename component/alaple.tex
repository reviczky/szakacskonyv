\project project
\product product

\startcomponent alaple

\startsubsection[title={Alaplé},bookmark={Alaplé},reference={alaple}]
\startitemize[R]
  \startitem zöldséglé: {\tt \unit{150 gram} sárgarépa, \unit{100 gram} gyökér, \unit{40 gram} hagyma} \crlf
             sárgarépa, zellergumó, édeskömény, póréhagyma, szárított gomba, fokhagyma, vöröshagyma, paradicsom, chili, paprika, petrezselyem, kakukkfű, rozmaring, egész fekete bors, babérlevél \stopitem
  \startitem csontlé: {\tt \unit{400 gram} csont} \stopitem
  \startitem csirkealaplé: {\tt \unit{2-2,5 kilogram} csirkecsont, csirke aprólék, esetleg csontos csirkemell, 3 db nagyobb sárgarépa, 2 db nagy fej vöröshagyma, 2 db fehérrépa, 5 gerezd fokhagyma, 1 csokor zellerzöld, 1 db babérlevél, 1 csokor petrezselyem 1 tk szemesbors. Esetleg: póréhagyma, paradicsom, TV paprika, szerecsendió virág} \stopitem
  \startitem hallé: {\tt az alaplébe (az amur kivételével) bármilyen hal jó, ha lehet legalább 5-6 féle \unit{400 gram} hal} \crlf
             Pikkelyezzük le, vágjuk le uszonyait, ezeket dobjuk ki. A májat is dobjuk ki! Ezután nagyon meleg vízben alaposan súroljuk meg a halakat, pl. körömkefével, hogy az összes nyálkát eltávolítsuk. Ettől romlik meg a halászlé. \crlf
             A fejek következnek. Az alsó állkapcsot vágjuk ketté, így könnyen kiszedhetjük a kopoltyúkat, a keserűcsontot és a szemeket is. Alapos mosás után ez is mehet a kondérba a kiszúrt úszóhólyagokkal együtt. A tetejére aprítunk kilónként két közepes fej hagymát, ha lehet kétfélét, majd annyi vízzel töltjük fel, amennyi teljesen ellepi, aztán forraljuk. Ha a víz már buzog, beledobjuk a pirospaprikát, kilónként úgy egy púpozott evőkanálnyit. Legalább kétféle legyen ez is. Ha akarjuk, dobhatunk bele laskára vágott zöldpaprikát is, de cseresznyepaprikát ne! Addig főzzük, míg a halak teljesen szét nem esnek, illetve a halhús le nem válik a csontokról. Ízlés szerint sózzuk. Majd egy nagyon aprólyukú húsdarálón letekerjük a maradékot. Ekkor az Y szálkákat is úgy feldaraboljuk, hogy fogyasztáskor semmi problémát nem okozhatnak. A darálékot belekeverjük az alaplébe. Ha a hal mocsárízű, mustár hozzáadásával ez az íz csökkenthető. \stopitem
\stopitemize
\stopsubsection

\stopcomponent
