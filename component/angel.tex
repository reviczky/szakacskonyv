\project project
\product product

\startcomponent angel

\startsubsection[title={Angyaltorta},bookmark={Angyaltorta},reference={angel}]
{\tt \unit{90 gram} rétesliszt, \unit{30 gram} kukoricaliszt, \unit{300 gram} porcukor, 12 (\unit{375 gram}) tojásfehérje, csipet só} \crlf
{\tt 10 tojásfehérje, 1 teáskanál borkősav vagy sütőpor, \unit{3 deciliter} porcukor, \unit{2,4 deciliter} liszt, $1/4$ teáskanál só, 1 teáskanál vaníliás cukor} \crlf
Kochen Amerika: {\tt 1 cup liszt (\unit{125 gram}, 12 ek), 1$1/2$ cup porcukor (\unit{375 gram}, 18 ek), 2 cup tojásfehérje (13 db, \unit{5 deciliter}), 1$1/2$ teaspoon cream of tartar borkősav, 1 teaspoon vanilleextract, $1/2$ teaspoon só} \crlf
A porcukor felét a liszttel háromszor átszitáljuk. A fehérjéket felverjük a porcukor felével, melyet kanalanként adunk a habhoz, vaníliás cukor, borkősav, só hozzá. A liszt‒cukor keveréket három részben adjuk hozzá. Rögtön sütjük a speciális formában. \unit{190 degree celsius} \unit{45 arcminute} felfordítva tíz percig hűlni hagyjuk. {\it Boston 669 Angel Food Cake}
\stopsubsection

\stopcomponent
