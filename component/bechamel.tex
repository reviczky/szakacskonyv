\project project
\product product

\startcomponent bechamel

\startsubsection[title={Bèchamel sauce},bookmark={Bèchamel sauce},reference={bechamel}]
\startitemize[R]
  \startitem Bèchamel sauce \unit{1500 gram}: {\tt \unit{140 gram} vaj, \unit{140 gram} liszt, \unit{1 liter} tej, só, szerecsendió} \crlf
             Világos vajas rántást készítünk, keverés mellett forró tejjel felengedjük, simára keverjük. Kis tűzön $1/2$ óráig egyenletesen forraljuk. Ízesítjük. Átszűrjük, és a felületét egy darab vajjal átsimítjuk, hogy ne bőrösödjön. \stopitem
  \startitem panádli: {\tt \unit{30 gram} vaj, \unit{60 gram} liszt, \unit{1 deciliter} víz} \crlf
             A vajat a vízzel felforraljuk, \unit{60 gram} lisztet belekeverjük, és ha az edénytől elvált, hidegre tesszük. Ízesítjük. \stopitem
  \startitem Bèchamel: {\tt \unit{200 gram} vaj, \unit{250 gram} liszt, \unit{6 deciliter} tej, \unit{10 gram} só} \stopitem
  \startitem Bèchamel: {\tt \unit{150 gram} zsír, \unit{250 gram} liszt, \unit{7 deciliter} tej, \unit{10 gram} só} \stopitem
  \startitem Sonkahab: {\tt \unit{600 gram} darált sonka, \unit{300 gram} bèchamel, \unit{250 gram} habosra kevert vaj} \stopitem
\stopitemize
\stopsubsection

\stopcomponent
