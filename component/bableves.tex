\project project
\product product

\startcomponent bableves

\startsubsection[title={Bableves, sólet, ricset},bookmark={Bableves, sólet, ricset},reference={bableves}]
{\tt \unit{250 gram} bab, \unit{700 gram} sertéscsülök (vagy más füstölt hús, kolbász vagy marhahús), \unit{200 gram} fehérrépa, \unit{100 gram} sárgarépa, 1 ek olaj, 2 ek liszt, 1‒2 gerezd fokhagyma, 1 mk pirospaprika, só, \unit{2 deciliter} tejföl, csipetke} \crlf
\unit{250 gram} babot előző napon langyos vízbe áztatjuk. Az áztatólevet leöntjük. Másnap \unit{1$1/2$ liter} hideg vízbe főni tesszük fel a csülköt. Két óráig is. Amikor a csülök félig puha, beletesszük a babot, ha a bab is félig meg megfőtt, hozzáadjuk a megtisztított, elnegyedelt zöldséget. Ha teljesen megpuhult a bab, berántjuk. A rántásba kevés reszelt hagymát, összezúzott foghagymát és pirospaprikát teszünk. Felengedjük \unit{2‒3 deciliter} hideg vízzel, simára keverjük és a levesbe öntjük. Felforraljuk és még legalább 10 percig főzzük. Sózzuk, ecettel savanyítjuk és babérlevéllel, pici cukorral édesítjük. Csipetkét főzünk bele. Tálaláskor tejföllel ízesítjük.
\stopsubsection

\stopcomponent
